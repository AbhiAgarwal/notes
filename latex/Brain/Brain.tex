\documentclass[11pt, oneside]{article}   	% use "amsart" instead of "article" for AMSLaTeX format
\usepackage{geometry}                		% See geometry.pdf to learn the layout options. There are lots.
\geometry{letterpaper}                   		% ... or a4paper or a5paper or ... 
%\geometry{landscape}                		% Activate for for rotated page geometry
%\usepackage[parfill]{parskip}    		% Activate to begin paragraphs with an empty line rather than an indent
\usepackage{graphicx}				% Use pdf, png, jpg, or eps§ with pdflatex; use eps in DVI mode
								% TeX will automatically convert eps --> pdf in pdflatex
\usepackage{amssymb}

\title{The Brain}
\author{Abhi Agarwal}
%\date{}							% Activate to display a given date or no date

\begin{document}
\maketitle
%\section{}
%\subsection{}

\section{''Brain''}

\par The brain is an Electrician's Nightmare.

\par There is an input coming in of some stimulus into the brain, and there's a black box (the brain) that does some processing, and gives an output. Neuroscience is about the black box in the middle.

\par Being human and being intelligent are different things.

\par Aim: An artificial system that uses the same functional architecture as an intelligent, living brain should be, likewise intelligent-and not just cont contrivedly so, but actually, truly intelligent. (Intelligence, 37)

\par In 1978 Mountcastle pointed out, in his paper ``An Organizing Principle for Cerebral Function", that the neocortex is remarkably uniform in appearance and structure. The regions that handle auditory input look like the regions that handle touch, which look like the regions that control muscles, which look like Broca's language area, which looks like particularly every other region of the cortex. He proposes that the cortex uses the same computational tool to accomplish everything it does, and perhaps they are actually performing the same basic operation.

\par A single learning algorithm can be created. The processing for the input can be done differently, but the middle process can be done with the same algorithm regardless of the sense.

\section{''Sources''}

- Book: On Intelligence, Jeff Hawkins
\newline - Video: Introduction to Neuroscience I and II, Stanford: 
\newline \indent Neuroscience I: http://www.youtube.com/watch?v=5031rWXgdYo
\newline \indent Neuroscience II: http://www.youtube.com/watch?v=uqU9lmFztOU

\end{document}  