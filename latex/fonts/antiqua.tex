\documentclass{article}
\usepackage{microtype}
\usepackage[T1]{fontenc}
\usepackage{antiqua}
\usepackage[utf8]{inputenc}
\pagestyle{empty}
\linespread{1.2}
\begin{document}
\frenchspacing
\noindent
{\LARGE 8 URW Antiqua}\\
~\\
Are you ready for a little \LaTeX{} magic? Type the following into a file named \mbox{\emph{antiqua.sty}}:

\begin{verbatim}
\NeedsTeXFormat{LaTeX2e}
\ProvidesPackage{antiqua}
\renewcommand{\rmdefault}{uaq}
\renewcommand{\sfdefault}{ugq}
\endinput
\end{verbatim}

\noindent
Now simply \textbackslash{}usepackage\{antiqua\}, and your document
will be typeset in URW Antiqua! Antiqua is little known, but still an
original creation by Hermann Zapf. Its sans serif counterpart that
will also be loaded by antiqua.sty is called \mbox{{\sf UWR Grotesk}}. Both
fonts were a commercial failure in the 1980s and were later made
available under the GNU General Public License.  The intention of this
typeface was to be a highly legible, classical font for use in books,
magazines, and newspapers.

Don't confuse this underrated beauty with ``Book Antiqua'', which is
Microsoft's knock-off of Zapf's Palatino typeface. (You know, the one
that's also known as Palladio by URW and Pagella by the \TeX{} Gyre
Project.)

\end{document}
